\documentclass{scrartcl}
\usepackage[T1]{fontenc}
\usepackage[latin1]{inputenc}
\usepackage[ngerman]{babel}
\usepackage{amsmath}
\usepackage{amsfonts}

\begin{document}
\section*{Aufgabe 1}
(a) $G=\{0\}$\\
(b) $B=C=()$\\
(c) $\Phi(0)=0$\\
(d) 0 (siehe (a)-(c))\\
(e) als ungenaue Antwort reicht $\infty$.\\
Eine genaue Antwort ist abz\"ahlbar unendlich.\\
Denn: $|G|>7 \Rightarrow \dim V \geq 1$. Es gilt, dass f\"ur jeden Basisvektor b aus V zwei Basisvektoren aus W ben\"otigt werden (z.B. b und ib).\\
$\Rightarrow \dim V = 2\cdot\dim W$. Dies ist bei endlich-dimensionalen Vektorr\"aumen nur f\"ur $\dim V = 0$ erf\"ullt.\\
Daher muss $|B|$ mindestens abz\"ahlbar sein. Dies reicht auch ($G=\mathbb{C}[X],B=(1,X,X^2,\dots),C=(1,i,X,iX,X^2,iX^2,\dots)$ und $\Phi(1)=1,\Phi(X)=i,\Phi(X^2)=X,\Phi(X^3)=iX,\dots$).
\end{document}
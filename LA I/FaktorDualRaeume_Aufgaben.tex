\documentclass{scrartcl}
\usepackage[T1]{fontenc}
\usepackage[latin1]{inputenc}
\usepackage[ngerman]{babel}
\usepackage{amsmath}
\usepackage{amsfonts}

\begin{document}
\section*{Aufgabe 1}
Sei K ein K�rper. $V := K[X]$ sei ein K-Vektorraum. Sei $U := [1] = [1+0X+0X^2+\dots] = [(1,0,0,\dots)]$.\\
\\
(a)	Gib eine Basis B von $^{K[X]}/_U$ an!\\
(b)	Gib einen Isomorphismus f von $^{K[X]}/_U$ nach K[X] an! Zeige auch: f ist wohldefiniert!\\
(c)	Gib $B^*=(g_1,g_2,g_3,\dots)$ $($die duale Vektorenmenge von B; f�r unendlichdimensionale Vektorr�ume ist $B^*$ keine Basis$)$ an! Zeige: $g_1,g_2,\dots$ sind wohldefiniert!\\
(d)	Sei $v=(2,5,3,6,7,0,0,\dots) + U$. Berechne $g_1(v), g_2(v), g_3(v), \dots$


\section*{Aufgabe 2}
Sei $V := {F_3}^3, B=\left(\begin{pmatrix}2\\1\\2\end{pmatrix},\begin{pmatrix}1\\2\\2\end{pmatrix},
\begin{pmatrix}1\\1\\1\end{pmatrix}\right)=(b_1,b_2,b_3)$.\\
(a) Zeige: B ist Basis von V.\\
(b) Berechne $B^*$.\\
(c) Sei $\Phi \colon V \rightarrow V^*,\Phi(b_i)=b_i^*$. Sei $x:=\left(\begin{pmatrix}2\\2\\1\end{pmatrix}\right)$. Berechne
$\varphi:=\Phi(x)$ und $\varphi(x)$.
\end{document}
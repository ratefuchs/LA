\documentclass{scrartcl}
\usepackage[T1]{fontenc}
\usepackage[latin1]{inputenc}
\usepackage[ngerman]{babel}
\usepackage{amsmath}

\begin{document}
\section*{Aufgabe 1}
Zeilenumformungen: $(1)-(2)\rightarrow(1), (2)-(3)\rightarrow(2), \dots, (n-1)-(n)\rightarrow(n-1)$ \\
Vebleibende Matrix:
$\begin{pmatrix} 1 & -1 & -1 & \cdots & -1 \\
							 1 & 1 & -1 & \cdots & -1 \\
							 1 & 1 & 1 & \cdots & -1 \\
							 \vdots & \vdots & \vdots & \ddots & \vdots \\
							 1 & 2 & 3 & \cdots & n \\
\end{pmatrix}$ \\
Spaltenumformungen: $(2)+(1)\rightarrow(2), (3)+(1)\rightarrow(3), \dots, (n)+(1)\rightarrow(n)$ \\
Vebleibende Matrix:
$\begin{pmatrix} 1 & 0 & 0 & \cdots & 0 \\
							 1 & 2 & 0 & \cdots & 0 \\
							 1 & 2 & 2 & \cdots & 0 \\
							 \vdots & \vdots & \vdots & \ddots & \vdots \\
							 1 & 3 & 4 & \cdots & n+1 \\
\end{pmatrix}$ \\
Determinante einer Dreiecksmatrix $=$ Produkt der Diagonaleintr�ge \\
$\Rightarrow \det A_n = 2^{n-2}\cdot (n+1)$ \\
$($Beachte: die Herleitung der Formel gilt nur f�r $n \geq 2$, die Formel selbst f�r alle n. F�r $n = 1$ �berpr�fung der Formel durch explizites ausrechnen$)$

\end{document}
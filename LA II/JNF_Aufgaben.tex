\documentclass{scrartcl}
\usepackage[T1]{fontenc}
\usepackage[latin1]{inputenc}
\usepackage[ngerman]{babel}
\usepackage{amsmath}
\usepackage{amsfonts}

\begin{document}
\section*{Aufgabe 1}
Berechne f�r jeden der K�rper $K=F_2, K=F_3$ und $K=\mathbb{R}$:\\
(a)	Falls vorhanden, die JNF $\tilde A$ zu A mit $A=\begin{pmatrix}1&2&2\\0&1&2\\0&1&2\end{pmatrix}$.\\
(b)	Gib, falls $\tilde A$ existiert, eine regul�re Matrix S an, f�r die $S^{-1}\cdot A\cdot S=\tilde A$ ist.

\section*{Aufgabe 2}

(a)	Begr�nde kurz: eine obere oder untere Dreiecksmatrix hat immer eine Jordannormalform.\\
(b)	Finde genau die $(2 \times 2)$-Matrizen �ber $F_2$, die KEINE JNF besitzen. Zeige die Korrektheit der L\"osung.\\
(c)	W�hle eine dieser Matrizen aus. Untersuche, ob die JNF existiert, wenn diese Matrix als Matrix in $F_3$ bzw. $F_5$ interpretiert wird, und gib diese gegebenenfalls an.\\
(d)	W�hle jeweils einen hier noch nicht genannten K�rper, dass die Matrix interpretiert als Matrix �ber diesem K�rper eine JNF besitzt bzw. nicht besitzt. Gib die JNF f�r den ersten Fall an.

\end{document}